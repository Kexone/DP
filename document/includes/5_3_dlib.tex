\subsection{Knihovna Dlib}
Autorem této knihovny je Davis King. Jedná se o~moderní C++ knihovnu obsahující algoritmy pro strojové učení a~nástroje pro vytváření komplexních programů v~jazyce C++. Používá se jak v~industriální, tak v~akademické sféře v~široké škále oblastí, jako jsou zejména vestavěná zařízení, robotika, mobilní telefony a~velká, výkonná výpočetní prostředí. Jedná se o~sbírku nezávislých softwarových komponent, kde každá z~nich je doprovázena důkladnou dokumentací a~mnoha příklady použití.

Jádrem filozofie této knihovny je věnování se snadnému používání a~přenositelnosti. Proto je kód navržen tak, aby nebylo po uživateli vyžadováno cokoli ručně konfigurovat nebo instalovat. K~dosažení tohoto cíle je veškerý kód specifický a~pro konkrétní platformu omezený a~obalený pomocí API rozhraní. Všechno ostatní je buď navrstveno na těchto obalech nebo napsáno v~normě ISO standardu C++. 

Knihovna se stále rozrůstá hlavně díky dobrovolným přispěvovatelům a~v~době psaní práce obsahuje například i~softwarové komponenty pro práci se sítí, vlákna, grafické rozhraní, komplexní datové struktury, lineární algebru, statistické strojové učení, zpracování obrazu, data mining, XML a~parsování textu, numerickou optimalizaci, Bayeské sítě. V~uplynulých letech byla velká část vývoje zaměřena na širokou sadu nástrojů statického strojového učení, avšak knihovna zůstává univerzální.  

V~současné době je známo, že knihovna pracuje na systémech OS X, MS Windows, Linux, Solaris, BSD, HP-UX. Knihovna by měla také pracovat na libovolné platformě POSIX, ale není otestovaná na všech dostupných verzích. 
Z~této knihovny byl v~práci použit pouze FHOG detektor objektů.
%Knihovna Dlib je vyvíjena primárně Davidem Kingem, který je jejím autorem a její počátky tkví již v roce 2002. Tato knihovna je otevřená, multiplatformní a navržena designově na zakázku a komponenty jsou založeny na softwarovém inženýrství, jedná se o sbírku nezávislých softwarových komponent z nichž je každá doprovázena důkladnou dokumentaci a mnoha příklady použítí.


