\section{Závěr}
Dle zadání práce byly otestovány vybrané metodiky z~knihoven OpenCV a~Dlib pro detekci chodců. Z~těchto metodik jsem vybral histogram orientovaných gradientů v~kombinaci s~lineárním klasifikátorem SVM z~OpenCV a~z~knihovny Dlib také histogram orientovaných gradientů. Použití těchto algoritmů obsahuje implementovaná aplikace. Lineární klasifikátor byl zvolen nejen díky své rychlosti klasifikace lineární dělící nadrovinou, ale také proto, že toto jádro je podporováno v~metodě s~posuvným oknem, tím pádem nebylo nutné vytvářet vlastní implementaci této detekce, která by jistě nedosahovala tak velkého výpočetního výkonu. 

Při trénování klasifikátoru hrálo důležitou roli správné zvolení trénovací sady a~trénovacích parametrů. Trénovací sada by měla být co nejkvalitnější a~měla by obsahovat co nejméně stínů a~artefaktů. Jak je zmíněno v~textu, aplikace disponuje křížovou validací a~testováním klasifikátoru, které sloužilo ke zvolení co nejvíce optimálních parametrů a~sady tak, abych docílil co nejpřesnějšího a~optimálního klasifikátoru.

Zjistil jsem, že má implementace dané problematiky detekce chodců na embedded zařízeních není dostačující. Na druhou stranu metoda substrakce pozadí značně urychlila detekci, a to minimálně na jeden snímek za sekundu.

Aplikace by mohla být v~budoucnu obohacena o~vlastní implementaci substrakce pozadí nebo jiným způsobem ořezání výstupního obrazu této substrakce, což by mohlo značně zvýšit výkon aplikace. Dále by práce mohla být rozšířena o~další klasifikátory, například kaskádovými. Tyto kaskádové klasifikátory by značně zvýšily výkon detekce chodců na daných zařízeních a~dosahovaly by většího počtu snímků za sekundu. Dalším možných vývojem je zvolit ARM zařízení s~grafickým jádrem a~implementovat histogram orientovaných gradientů za pomocí technologie Cuda. Také předběžné zpracování obrazu před samotnou detekcí by mohlo být provedeno paralelně.

%Dle zadání práce byly otestované vybrané metodiky z knihoven OpenCV a Dlib pro rozpoznávání chodců. Z těchto metodik byly vybrány algoritmy z OpenCV - Histogram orientovaných gradientů, kaskádové klasifikátory a z Dlib knihovny byl vybrán Histogram orientovaných gradientů a tyto algoritmy byly implementovány v aplikaci. Obě tyto knihovny disponují nástroji pro natrénování klasifikátoru, které se dají poté použít na samotnou detekci.

%Trénování klasifikátoru hraje důležitou roli pro samotnou detekci, protože zvyšuje přesnost detekce a účinnost klasifikátoru.  Další významnou roli na samotnou úspěšnost klasifikátoru je množina trénovacích dat a nastavení trénování klasifikátoru. Tato data by měla být co nejpřesnější a obsahovat minimum stínu a minimum artefaktů. Jak bylo již v tomto dokumentu zmíněno, aplikace obsahuje i nástroje pro otestování klasifikátoru na daných testovacíh datech, které slouží především k přibližnému zvolení parametrů klasifikátoru.

%Tato aplikace by mohla být nadále rozšiřována dalšími klasifikátory pro porovnání rychlosti a výsledkům k již naimplementovaným. 