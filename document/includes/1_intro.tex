\section{Úvod}
Detekce chodců představuje velmi náročný úkol, který v~posledních letech přitahuje velkou pozornost. 
Aplikace tohoto typu mohou mít široké uplatnění jak v~osobním, tak v~industriálním využití. Může se jednat o~bezpečnostní prvky, například na letištích, kde program může sledovat pohyb daného chodce a~vyhodnocovat tak jeho chování. Případně jako kamera průmyslového vozidla, kde řidič může přehlédnout chodce v~blízkosti vozidla, a~předejít tak neštěstí, kdy program může zamezit pohybu vozidla. Dalšími příklady mohou být detekce chodců na přechodech pro chodce, v~továrnách, kde se může pomocí rozšíření algoritmu o~rozpoznání obrazu vyhodnocovat chování a~docházka zaměstnanců. 

Nutno podotknout, že detekce chodců v~obrazech není pro lidské oko tak obtížným úkolem a~dokážou bez větší námahy rozpoznat všechny osoby v~obraze. Pro stroje je naopak tento úkol velkou výzvou. Chodci v~obrazech mohou mít různý tvar těla, barvu kůže, jiný postoj, také různý počet vrstev a~barev oblečení na sobě. Mohou být také z~části zakrytí nějakým objektem v~obraze, který nepodléhá samotné detekci, což může způsobit záporné vyhodnocení. Také zde má velký vliv vzdálenost osoby od kamery, kdy se pro strojovou doménu může stát chodec nedetekovatelný. 

Velkou zásluhu na strojovém detekování chodců má Navneet Dalal a~Bill Triggs, kde ve své práci \cite{hog:dalal} pomocí metodiky histogramu orientovaných gradientů úspěšně detekují chodce v~obrazech. Tato práce se z~větší části inspiruje tímto dokumentem a~rozšiřuje jej o~substrakci pozadí, která má za úkol urychlit a~zlepšit samotnou detekci chodců v~obrazech. Principem tohoto algoritmu je spustit detekci pouze v~oblastech obrazu, které se v~čase mění a~mohou představovat chodce. Ovšem tento typ algoritmu lze použít pouze na videosnímcích, které jsou pořízené ze statické kamery.

Hlavním cílem této práce je optimalizovat a~otestovat detektor chodců pro počítače s~architekturou ARM a~využít tak jejich značný výkon. 
Experimenty této práce se zaměří jak na trénování a~testování klasifikátoru s~různými parametry, tak na detekci chodců na samostatných snímcích, ale také i~ve videosekvencích, které budou detailně zaznamenány a~vyhodnoceny. 
V~druhé kapitole budou představeny hlavní výzvy a~problémy detekce chodců v~obrazech.
Jedna z~kapitol se bude věnovat metodikám pro detekci chodců a~využitých knihoven pro zpracování obrazu, které byly v~této práci použity.
V~dalších kapitolách bude popsána implementace programu, popis jeho funkcí, kterými tato aplikace disponuje. Také zde budou uvedena a~popsána zařízení, na kterých byl algoritmus otestován. V~neposlední řádě experimenty a~jejich dosažené výsledky.