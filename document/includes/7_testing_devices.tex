\section{ARM zařízení}
ARM systémy jsou kompletní počítače, vybudované na jedné desce s~procesorem, pamětí, vstupy, výstupy a~dalšími funkcemi. Jednodeskové počítače (\textit{SBC} - single-board computer) byly vyrobeny za účelem vývoje aplikací nebo jejich demonstraci, také slouží jako systémy pro vzdělávání nebo jsou použity jako vestavěné počítačové kontrolery. Typickým využitím těchto malých počítačů pro domácnost je její automatizace. 

Mezi známé a~výkonné můžeme zařadit následující vestavěné systémy.

\subsubsection*{Nvidia JETSON TK1}
Tento embedded systém disponuje nejen čtyřjádrovým procesorem Cortex-A15, ale i~grafickým čipem Nvidia Kepler s~192 Cuda jádry. Tato kombinace je jinak nazvaná jako Tegra K1 SoC (System on a~chip – integrovaný v~jediném obvodě). Tento procesor je údajně o~50\% výkonnější než Cortex-A9\footnote{https://developer.arm.com/products/processors/cortex-a/cortex-a15} a~podporuje instrukční sadu Armv7-A.

\subsubsection*{Asus Tinker Board}
Asus Tinker Board se řadí podle specifikace mezi ty výkonnější systémy. Na této desce se nachází čtyřjádrový procesor Cortex-A17 s~možností dynamického přetaktování (Turbo-Boost) až na 2,6 GHz. Tento počítač také podporuje instrukční sadu Armv7-A.  

\subsubsection*{Rock64}
Rock64 je produktem firmy Pine64 a~je osazen čtyřjádrovým procesorem Cortex-A53 a~až 4 GB operační paměti. Dále vyniká s~konektivitou USB 3.0 a~128 MB sériovou flash pamětí. Procesor podporuje instrukční sadu Armv8-A.

\subsubsection*{Odroid-XU4}
Tento počítač je vybaven osmijádrovým mobilním procesorem Samsung Exynos 5422. Tento procesor nabízí čtyřjádrový Cortex-A15 s~taktem 2,1 GHz a~další čtyřjádrový Cortex-A7 s~taktem 1,4 GHz. Což dělá tento počítač velmi výkonným na paralelní práci. Opět tento procesor podporuje instrukční sadu Armv7-A.

\subsubsection*{Intel Galileo}
Intel Galileo je prvním nízkoodběrovým produktem této firmy, disponuje jednojádrovým procesorem Intel Quark SoC X1000 s~taktem 400 MHz. První generace byla dostupná již v~roce 2013. Deska je navržena pro IoT (\textit{Internet of Things}) a~je zcela kompatibilní s~produkty, knihovny a~vývojovým prostředím elektrické platformy Arduino. 

\subsubsection*{Orange PI Plus 2E}
Orange PI je počítač velmi podobný Raspberry PI, ovšem svým výkonem je spíše obdobný počítači Banana PI. Orange PI Plus 2E je vybaven čtyřjádrovým procesorem Cortex-A7 s~podporou instrukční sady Armv7-A, 2 GB operační pamětí nebo například WiFi modulem s~anténou. Výhodou tohoto počítače je využití standardního DC konektoru k~napájení.

\subsubsection*{Cubieboard 5}
Tento počítač je osazen SoC čipem Allwinner H8. Jedná se o~symetrické osmijádro složené z~Cortex-A7 o~frekvenci 2,0 GHz. Dále na desce můžeme najít 2 GB operační paměti a~8 GB vestavěné úložiště. Tento embedded systém se od ostatních výše zmíněných liší tím, že na desce má osazený audio konektor S/PDIF a~DisplayPort, což umožnuje připojení až dvou monitorů a~také podporou RAID polí díky rozšiřující kartě. 

\subsection{Použitá zařízení}
K~otestování mého programu jsem využil počítače architektury ARM. Jedná se o~počítače SolidRun HummingBoard Pro, Raspberry PI 3 Model B a~Sinovoip Banana PI BPI-M1. Na těchto zařízeních bude otestován algoritmus a~všechny výsledky zapsány do tabulky v~následující kapitole.

\subsubsection*{HummingBoard Pro}
 Disponuje čtyřjádrovým procesorem i.MX6 Dual-core Lite na achitektuře Cortex A9 o~frekvenci 1 GHz, na desce má osazenou paměť typu DDR3 o~velikosti 2~GB a~nabízí grafický obvod Vivante GC880. Dále deska nabízí mSata konektor pro připojení SSD disku nebo například IR přijímač. Podporuje známé operační systémy Linux, například Android 4.4, Debian a~OpenSuse. Spotřeba zařízení je 2 W (0,41 A) v~klidovém stavu a~5 W (1 A) v~zátěži.

\subsubsection*{Raspberry PI 3}
Jedná se o~třetí generaci velmi úspěšné řady Raspberry PI. Model je osazen výkonným 64 bitovým čtyřjádrovým procesorem Cortex-A53 o~frekvenci 1,2 Ghz, grafickým čipem Broadcom VideoCore IV o~frekvenci 400 MHz, operační paměť typu SDRAM o~velikosti 1GB, která je sdílená s~grafickým čipem. Od svých předchůdců se líší integrovaným Wifi čipem podporující protokoly 802.11 b/g/n a~Bluetooth 4.1 LE. Díky architektuře ARMv8 má šiřší podporu Linuxů, včetně mobilního systému Android a~Windows 10 IoT. Jeho výkon bez zátěže je 1,5 W (300 mA) a~při zátěži se zapojenými periferiemi maximálně 6,7 W (1,34 A).

\subsubsection*{Banana PI}
Tento počítač je osazen dvoujádrovým procesorem AllWinner A20 s~frekvencí 1,2 GHz. Jedná se o~low-end verzi procesoru AllWinner A31. Dále na desce najdeme grafický čip Mali-400 MP2 a~operační paměť 1 GB. Jedná se o~první klon RPI a~jeho deska je osazena navíc například sata konektorem, mikro-USB OTG nebo mikrofonem. Tento mikropočítač díky GLAN můžeme použít jako vzdálené NAS úložiště (\textit{Network Attached Storage}), databázi, mail server nebo například web server. Jeho spotřeba v~nečinném stavu je 1,75 W (350 mA), maximálně 5.5 W (1,1 A).

\subsection{Srovnání použitých zařízení}
Jedná se o~poměrně výkonné počítače vzhledem ke své velikosti. V~následující tabulce se nachází srovnání některých parametrů těchto počítačů.
\begin{table}[H]
\centering
\caption{Srovnání testovaných zařízení}
\begin{tabular} { |c|c|c|c| }
\hline
{}                  & {HummingBoard Pro}    & {Raspberry PI 3}      & {Banana PI}          \\ \hline
Procesor            & NXP i.MX6 ARM         & ARM                   & A20 ARM              \\ \hline
Počet jader         & 4                     & 4                     & 2                    \\ \hline
Kmitočet            & 1 GHz                 & 1,2 GHz               & 1,2 GHz              \\ \hline
Druh architektury   & Cortex A9             & Cortex A53            & Cortex A7            \\ \hline
Grafický čip        & Vivante GC880         & BroadCom VideoCore IV & Mali-400 MP2         \\ \hline
Kapacita paměti RAM & 1 GB                  & 1 GB                  & 1 GB                 \\ \hline
Druh paměti         & DDR3                  & LPDDR2                & DDR3                 \\ \hline
Ethernet            & 10/100/1000           & 10/100                & 10/100/1000          \\ \hline
Počet USB portů     & 2                     & 4                     & 2                    \\ \hline
Úložný prostor      & MicroSD               & MicroSDHC             & SD/MMC               \\ \hline
Rok vydání          & 2014                  & 2016                  & 2014                 \\ \hline
\end{tabular}
\label{srovnaniPC}
\end{table}
