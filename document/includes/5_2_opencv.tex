\subsection{Knihovna OpenCV}
OpenCV (\textit{Open Source Computer Vision Library}) je open-source knihovna, která slouží k~zpracování obrazu a~strojového učení. Je psaná v~optimalizovaném C/C++ jazyce s~podporou multijádrového zpracování.

Knihovna má více než 2500 optimalizovaných algoritmů, které zahrnují obsáhlé sady klasických a~moderních algoritmů pro zpracování obrazu a~strojové učení. Tyto algoritmy mohou být použity na detekci a~rozpoznávání tváří, identifikování objektů, vyhodnocování lidských akcí ve videosekvencích, sledování pohybů pomocí kamery, sledování pohybujících se objektů, extrahování 3D modelů z~objektů, vytváření 3D mračen bodů ze stereo kamer, spojování obrazů a~vytvořit tak obraz s~vysokým rozlišením na dané scéně, hledání podobných obrazů z~obrazové databáze, odstranění červených očí z~obrazu způsobené bleskem, sledování pohybu očí, rozeznání scenérie a~vytvoření značek pro překrytí rozšířenou realitou a~další.

Knihovna má rozhraní pro jazyky C++, C, Python, Java a~MATLAB a~podporuje operační systémy Windows, Linux, Android, iOS a~MacOS. 
Také využívá MMX a~SSE instrukce, pokud jsou k~dispozici, což zvyšuje její výkon.

Podílí se na ní komunita lidí, kterou tvoří více než 47 tisíc uživatelů celého světa a~přesahuje více než 14 milionů stažení. 

%OpenCV leans mostly towards real-time vision applications and takes advantage of MMX and SSE instructions when available. A full-featured CUDA and OpenCL interfaces are being actively developed right now. There are over 500 algorithms and about 10 times as many functions that compose or support those algorithms. OpenCV is written natively in C++ and has a templated interface that works seamlessly with STL containers. 
%OpenCV (Open Source Computer Vision Library) is released under a BSD license and hence it’s free for both academic and commercial use. It has C++, C, Python and Java interfaces and supports Windows, Linux, Mac OS, iOS and Android. OpenCV was designed for computational efficiency and with a strong focus on real-time applications. Written in optimized C/C++, the library can take advantage of multi-core processing. Enabled with OpenCL, it can take advantage of the hardware acceleration of the underlying heterogeneous compute platform.

%Adopted all around the world, OpenCV has more than 47 thousand people of user community and estimated number of downloads exceeding 14 million. Usage ranges from interactive art, to mines inspection, stitching maps on the web or through advanced robotics.

Z~této knihovny byly použité následující nástroje:

\begin{itemize}
  \item{Mixtura Gausiánů,}
  \item{Konvexní obal,}
  \item{Histogram orientovaných gradientů,}
  \item{Support vector machines.}
\end{itemize}